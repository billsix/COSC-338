\documentclass[11pt,twocolumn]{article}
\usepackage{hyperref,color}
\usepackage[margin=.5in]{geometry}
\usepackage{titlesec}
\usepackage{longtable}
\usepackage{gb4e}
%\usepackage[margin=.75in]{geometry}

\pagenumbering{gobble}

%paragraph formatting
\setlength{\parindent}{0pt}
\setlength{\parskip}{7pt}

\titlespacing\subsection{0in}{\parskip}{\parskip}
\titlespacing\section{0in}{\parskip}{\parskip}

%Just fill these in to fill in the basic syllabus information.
\newcommand{\coursename}{COSC-338}
\newcommand{\semester}{Spring 2020}
\newcommand{\classtimes}{M 6pm-8pm, W 6pm-8pm}
\newcommand{\myname}{Bill Six}
\newcommand{\myemail}{wesix@smcm.edu}
\newcommand{\university}{St. Mary's College Of Maryland}

\title{\coursename}
\author{{\university}---{\semester}---{\classtimes}}
\date{}


\begin{document}
\maketitle


\section{Instructor information}

\begin{tabular}{ll}
Name:&\myname \\
Contact:&\href{mailto:\myemail}{\myemail}\\
\end{tabular}

\section{Course description}

%Add new sections and content as needed.
Introduction to Computer Graphics, using OpenGL, both in Python and in C++.
The course will cover creating applications in 2D and in 3D, geometric
transformations, color and lighting, imaging, and shadows.  The course
begins with an easy to understand version of OpenGL, version 2, and progresses
to version 3.3+, which allows the programmer to create much more realism
in the graphics.

\section{Grading}

\begin{center}
\begin{tabular}{cc}
\begin{tabular}{l|l}	%For grade items (quizzes, homework, etc.)
Item&Percent\\\hline\hline
Homework&35\\
Midterm&15\\
Final&20\\
Projects&15\\
Paper&15
\end{tabular}
&
\begin{tabular}{ll}

\end{tabular}
\end{tabular}
\end{center}

\section{Schedule}

TBD based off of feedback.  We will cover:


\begin{enumerate}
\item{modelviewprojection. github.com/billsix/modelviewprojection.  Covers the basics of
  OpenGL, and how to think about placing objects in a scene, how to place a camera, and
how to use basic inputs from the keyboard.}
\item{OpenGL SuperBible v4 github.com/billsix/COSC-338/.  A thorough explanation of graphics
concepts, OpenGL2, with a transition to OpenGL 3+}
\item{Week of Mar 23 - Chapter 5 - Colors, Materials, and Lighting}
\item{Week of Mar 30 - Chapter 8 and 9 - Texturing}
\item{Week of Apr 6 - Chapter 11 and 12 - Faster Geometry Throughput, Selection}
\item{Week of Apr 13 - Chapter 15 and 16 - Shader language.  Vertex Shaders}
\item{Week of Apr 20 - Chapter 16 and 17 - Vertex Shaders, Fragment Shaders.}
\item{Week of Apr 27 - Misc topics of interest}
\item{OpenGL SuperBible v4 github.com/billsix/COSC-338/.  A thorough explanation of graphics
concepts, OpenGL2, with a transition to OpenGL 3+}
\item{http://www.opengl-tutorial.org. A great resource for learning OpenGL 3.3+}
\item{This is a reading and modifying heavy class, not a write everything from scratch class.
  There will be small projects to demonstrate understanding of the material.}
\item{There will be a project in which you study the source of a project of your choice,
  and write a paper about what you learned, and/or modify it to do something different.
  For instance, there is an open source Minecraft clone
  https://github.com/fogleman/Craft, and open source Portal like game https://github.com/GlPortal/glPortal.
  }

\end{enumerate}

\section{Software/OS}


Modelviewprojection requireds Python3 to be installed, and works on Linux/Windows/macOS.
On Windows, install Visual Studio Community, and install the Python Extension tools.
On Mac, install Python through the regular installer, or anaconda, or through macports or homebrew.
On Linux, use the package manager.



\end{document}
